
% !TEX encoding = UTF-8 Unicode 
% !TEX root = on_gates.tex

\clearpage

\section{Standard 3-qubit gates}

\paragraph{Toffoli gate (controlled-controlled-not, CCNOT)}\cite{Toffoli1980a, Barenco1995b}

\[
\text{CCNOT} =
\begin{bsmallmatrix}
 1 & 0 & 0 & 0 & 0 & 0 & 0 & 0 \\
 0 & 1 & 0 & 0 & 0 & 0 & 0 & 0 \\
 0 & 0 & 1 & 0 & 0 & 0 & 0 & 0 \\
 0 & 0 & 0 & 1 & 0 & 0 & 0 & 0 \\
 0 & 0 & 0 & 0 & 1 & 0 & 0 & 0 \\
 0 & 0 & 0 & 0 & 0 & 1 & 0 & 0 \\
 0 & 0 & 0 & 0 & 0 & 0 & 0 & 1 \\
 0 & 0 & 0 & 0 & 0 & 0 & 1 & 0 \\
\end{bsmallmatrix}
\]
$$
\input{circuits/ccnot.tex}
$$

\newcommand{\sm}{{\text{-}}} % Small minus

\[
\ham{CCNOT} & = -\tfrac{\pi}{8}(I_0-Z_0)(I_1-Z_1)(I_2-X_2) \\
& =
\tfrac{\pi}{2} \begin{bsmallmatrix*}[r]
 0 & 0 & 0 & 0 & 0 & 0 & 0 & 0 \\
 0 & 0 & 0 & 0 & 0 & 0 & 0 & 0 \\
 0 & 0 & 0 & 0 & 0 & 0 & 0 & 0 \\
 0 & 0 & 0 & 0 & 0 & 0 & 0 & 0 \\
 0 & 0 & 0 & 0 & 0 & 0 & 0 & 0 \\
 0 & 0 & 0 & 0 & 0 & 0 & 0 & 0 \\
 0 & 0 & 0 & 0 & 0 & 0 & \sm 1 & 1 \\
 0 & 0 & 0 & 0 & 0 & 0 & 1 & \sm 1 \\
\end{bsmallmatrix*}
\notag
\]



\paragraph{Fredkin gate (controlled-swap, CSWAP)}\cite{Fredkin1982a,???}
\[
\text{CSWAP} = 
\begin{bsmallmatrix}
 1 & 0 & 0 & 0 & 0 & 0 & 0 & 0 \\
 0 & 1 & 0 & 0 & 0 & 0 & 0 & 0 \\
 0 & 0 & 1 & 0 & 0 & 0 & 0 & 0 \\
 0 & 0 & 0 & 1 & 0 & 0 & 0 & 0 \\
 0 & 0 & 0 & 0 & 1 & 0 & 0 & 0 \\
 0 & 0 & 0 & 0 & 0 & 0 & 1 & 0 \\
 0 & 0 & 0 & 0 & 0 & 1 & 0 & 0 \\
 0 & 0 & 0 & 0 & 0 & 0 & 0 & 1 \\
\end{bsmallmatrix}
\]

$$
\input{circuits/cswap.tex}
$$

\[
\ham{CSWAP} & = -\tfrac{\pi}{8}(I_0-Z_0)(X_1 X_2 + Y_1 Y_2 + Z_1 Z_2 - I_1 I_2)\\
& =
\tfrac{\pi}{2} \begin{bsmallmatrix*}[r]
 0 & 0 & 0 & 0 & 0 & 0 & 0 & 0 \\
 0 & 0 & 0 & 0 & 0 & 0 & 0 & 0 \\
 0 & 0 & 0 & 0 & 0 & 0 & 0 & 0 \\
 0 & 0 & 0 & 0 & 0 & 0 & 0 & 0 \\
 0 & 0 & 0 & 0 & 0 & 0 & 0 & 0 \\
 0 & 0 & 0 & 0 & 0 & \text{-}1 & 1 & 0 \\
 0 & 0 & 0 & 0 & 0 & 1 & \text{-}1 & 0 \\
 0 & 0 & 0 & 0 & 0 & 0 & 0 & 0 \\
\end{bsmallmatrix*}
\notag
\]

\paragraph{CCZ gate (controlled-controlled-Z)}
\[
CCZ=
\begin{bsmallmatrix*}[r]
 1 & 0 & 0 & 0 & 0 & 0 & 0 & 0 \\
 0 & 1 & 0 & 0 & 0 & 0 & 0 & 0 \\
 0 & 0 & 1 & 0 & 0 & 0 & 0 & 0 \\
 0 & 0 & 0 & 1 & 0 & 0 & 0 & 0 \\
 0 & 0 & 0 & 0 & 1 & 0 & 0 & 0 \\
 0 & 0 & 0 & 0 & 0 & 1 & 0 & 0 \\
 0 & 0 & 0 & 0 & 0 & 0 & 1 & 0 \\
 0 & 0 & 0 & 0 & 0 & 0 & 0 & \sm1 \\
\end{bsmallmatrix*}
\]
$$
\input{circuits/ccz.tex}
$$




\paragraph{Peres gate}~\cite{Peres1985a}
\[
\text{Peres} = 
\Left[ \begin{smallmatrix}
                1& 0& 0& 0& 0& 0& 0& 0 \\
                0& 1& 0& 0& 0& 0& 0& 0 \\
                0& 0& 1& 0& 0& 0& 0& 0 \\
                0& 0& 0& 1& 0& 0& 0& 0 \\
                0& 0& 0& 0& 0& 0& 0& 1 \\
                0& 0& 0& 0& 0& 0& 1& 0 \\
                0& 0& 0& 0& 0& 1& 0& 0 \\
                0& 0& 0& 0& 1& 0& 0& 0 
\end{smallmatrix} \Right]
\]
Another gate that is universal for classical reversible computing. It is equivalent to a Toffoli followed by a CNOT gate.

$$
\adjustbox{scale=0.8}{\begin{quantikz}[thin lines, column sep=0.75em,row sep={2.5em,between origins}]
& \ctrl{1} & \ctrl{1} & \qw \\
& \ctrl{1} & \targ{} & \qw \\
& \targ{} & \qw & \qw
\end{quantikz}
}
$$




\paragraph{Deutsch gate}~\cite{Deutsch1989a, Barenco1995a, Shi2018a}
 Mostly of historical interest, since this was the first quantum gate to be shown to be computationally universal~\cite{Deutsch1989a}. 
\[
\text{Deutsch}(\theta) =
\begin{bsmallmatrix}
                1& 0& 0& 0& 0& 0& 0& 0 \\
                0& 1& 0& 0& 0& 0& 0& 0 \\
                0& 0& 1& 0& 0& 0& 0& 0 \\
                0& 0& 0& 1& 0& 0& 0& 0 \\
                0& 0& 0& 0& 1& 0& 0& 0 \\
                0& 0& 0& 0& 0& 1& 0& 0 \\
                0& 0& 0& 0& 0& 0& i \cos(\theta) & \sin(\theta) \\
                0& 0& 0& 0& 0& 0& \sin(\theta)& i \cos(\theta)
\end{bsmallmatrix}
\]
Examining the controlled unitary sub-matrix, the Deutsch gate can be thought of as a controlled-controlled-$i\Gate{R_x}(\theta)^2$ gate. gate.
$$
\text{Deutsch}(\theta) = 
\input{circuits/Deutsch}
$$

Barenco~\cite{Barenco1995a} demonstrated a construction of the Deutsch gate from 2-qubit ``Barenco'' gates, demonstrating that 2-qubits gates are sufficient for universality.
$$
 \input{circuits/Deutsch}
 \simeq 
\adjustbox{scale=0.75}{\begin{quantikz}[thin lines, column sep=0.75em,row sep={2.5em,between origins}]
& \qw & \ctrl{2} & \ctrl{1} & \qw & \ctrl{1} & \qw \\
& \ctrl{1} & \qw & \gate{\text{Bar}(0, \tfrac{\pi}{2}, \tfrac{\pi}{2})} & \ctrl{1} & \gate{\text{Bar}(0, \tfrac{\pi}{2}, \tfrac{\pi}{2})} & \qw \\
& \gate{\text{Bar}(0, \tfrac{\pi}{4}, \tfrac{\theta}{2})} & \gate{\text{Bar}(0, \tfrac{\pi}{4}, \tfrac{\theta}{2})} & \qw & \gate{\text{Bar}(\pi, -\tfrac{\pi}{4}, \tfrac{\theta}{2})} & \qw & \qw
\end{quantikz}
} 
% \adjustbox{scale=0.8}{\begin{quantikz}[thin lines, column sep=0.75em,row sep={2.5em,between origins}]
& \qw & \ctrl{2} & \ctrl{1} & \qw & \ctrl{1} & \qw \\
& \ctrl{1} & \qw & \gate{\text{Bar}(0, \frac{\pi}{2}, \frac{\pi}{2})} & \ctrl{1} & \gate{\text{Bar}(0, \frac{\pi}{2}, \frac{\pi}{2})} & \qw \\
& \gate{\text{Bar}(0, \frac{\pi}{4}, \frac{\theta}{2})} & \gate{\text{Bar}(0, \frac{\pi}{4}, \frac{\theta}{2})} & \qw & \gate{\text{Bar}(\pi, - \frac{\pi}{4}, \frac{\theta}{2})} & \qw & \qw
\end{quantikz}
}
 $$


% Also cite: https://journals.aps.org/prapplied/abstract/10.1103/PhysRevApplied.9.051001
% for proposed physical implementation Shi2018a


% QFT: Quantum Fourier transform,=. Not to be confuesd wiht, in different conetxts Quantu Field Theory, or Quanutm Flutuation theorem~\cite{Theotherqft}. At least Q alwyas standas for Quantum.

