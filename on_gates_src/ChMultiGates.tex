
% !TEX encoding = UTF-8 Unicode 
% !TEX root = on_gates.tex

\clearpage

\section{Multi-qubit gates}

\subsection{Controlled unitary gates}

\subsection{Two-level unitary gates}

A 2-level unitary is a unitary operation that acts non-trivially on only 2-states. For 2-qubits, any controlled-unitary gate is 2-level,
\[
\begin{bsmallmatrix}
    1 & 0 & 0 & 0 \\ 
    0 & 1 & 0 & 0 \\ 
    0 & 0 & a & c \\ 
    0 & 0 & b & d        
\end{bsmallmatrix}
\]
where the 2x2 submatrix $\begin{bsmallmatrix}
    a & c \\ 
    b & d        
\end{bsmallmatrix}$ 
is unitary. But the active states need not be the last two. Any permutation of a controlled 1-qubit gate is also a two-level unitary, such as 
\[
\begin{bsmallmatrix}
    a & c & 0 & 0 \\ 
    b & d & 0 & 0 \\ 
    0 & 0 & 1 & 0 \\ 
    0 & 0 & 0 & 1        
\end{bsmallmatrix}
\text{or}
\begin{bsmallmatrix}
    a & 0 & 0 & c \\ 
    0 & 1 & 0 & 0 \\ 
    0 & 0 & 1 & 0 \\ 
    b & 0 & 0 & d        
\end{bsmallmatrix} \ .
\]
Similarly any multi-controlled 2x2 unitary, or permutation of the same, is a 2-level unitary. 

Two-level unitaries are important because we can systematically decompose an arbitrary unitary into a sequence of 2-level unitaries, and 2-level unitaries themselves can be decomposed into controlled-unitary gates. 

% Cite Reck1994
\subsection{Multiplexed gates}

\subsection{Diagonal gates}


